\documentclass[a4paper]{article}

\usepackage[margin=1in]{geometry}

\usepackage{blindtext}
\usepackage[utf8]{inputenc}
\usepackage[greek,english]{babel}
\usepackage{alphabeta}

\usepackage{inconsolata}
\usepackage{listings}
\usepackage{xcolor}
\lstset{
    language=C, 
    basicstyle=\ttfamily\small,
    numberstyle=\footnotesize,
    numbers=left,
    backgroundcolor=\color{gray!10},
    frame=single,
    tabsize=2,
    rulecolor=\color{black!30},
    title=\lstname,
    escapeinside={\%*}{*)},
    breaklines=true,
    breakatwhitespace=true,
    framextopmargin=2pt,
    framexbottommargin=2pt,
    inputencoding=utf8,
    extendedchars=true,
    literate={á}{{\'a}}1 {ã}{{\~a}}1 {é}{{\'e}}1,
}

\lstdefinelanguage
   [x64]{Assembler}     % add a "x64" dialect of Assembler
   [x86masm]{Assembler} % based on the "x86masm" dialect
   % with these extra keywords:
   {morekeywords={CDQE,CQO,CMPSQ,CMPXCHG16B,JRCXZ,LODSQ,MOVSXD, %
                  POPFQ,PUSHFQ,SCASQ,STOSQ,IRETQ,RDTSCP,SWAPGS, %
                  rax,rdx,rcx,rbx,rsi,rdi,rsp,rbp, %
                  r8,r8d,r8w,r8b,r9,r9d,r9w,r9b, %
                  r10,r10d,r10w,r10b,r11,r11d,r11w,r11b, %
                  r12,r12d,r12w,r12b,r13,r13d,r13w,r13b, %
                  r14,r14d,r14w,r14b,r15,r15d,r15w,r15b,
				  lw, sw, li, add, addi, csrw, lui, auipc, bgeu, 
				  bltu, jal, j, unimp, srli, andi, srai, beqz, jalr, bne,
				  bge, beq, blt}} % etc.

\lstset{language=[x64]Assembler}

\title{ΕΘΝΙΚΟ ΜΕΤΣΟΒΙΟ ΠΟΛΥΤΕΧΝΕΙΟ\\ ~\\Αναφορά Πρώτου Εργαστήριου RISC-V\\Μάθημα: Εργαστήριο Μικροϋπολογιστών}
\author{Ονοματεπώνυμο: Μάρκος Γκέργκες\\ Αριθμός Μητρώου: 03117870}
\date{Ακαδημαϊκό Έτος 2020-21 }
\begin{document}
\maketitle

\section*{Ερώτημα 1}
\par Για το διάβασμα των διακοπτών, και την απεικόνιση των leds, χρησιμοποίηθηκαν οι ίδιες συναρτήσεις που ορίστηκαν στα πλαίσια του μαθήματος. Πιο συγκεκριμένα, η READ\_GPIO 
είναι ένα macro το οποίο επιστρέφει το περιεχόμενο ενός δείκτη που παίρνει σαν παράμετρο. Παρόμοια, η WRITE\_GPIO θέτει το περιεχόμενο ενός δείκτη με μια τιμή. Γίνεται typecasting 
σε 32bit unsigned, ενώ η δεσμευμένη λέξη volatile διασφαλίζει ότι ο μεταγλωττιστής θα διαβάζει κάθε φορά την τιμή της μεταβλητής από τη μνήμη. Έτσι, και σε περίπτωση διακοπής, η 
τιμή της μετβλητής θα είναι η σωστή. 
\subsection*{Κώδικας C}
\begin{lstlisting}[language=C++, keywordstyle=\color{red}, basicstyle=\small]
// From memory-map
#define GPIO_SWs    0x80001400
#define GPIO_LEDs   0x80001404
#define GPIO_INOUT  0x80001408

//define basic read-write macros
#define READ_GPIO(addr) (*(volatile unsigned*)addr)
#define WRITE_GPIO(addr, value) { (*(volatile unsigned *)addr) = (value);}

int main(void) 
{
  volatile unsigned ddr_value=0xFFFF, msb_val, lsb_val, sum;

  WRITE_GPIO(GPIO_INOUT, ddr_value);							//set leds as output
  while (1) {
      msb_val = READ_GPIO(GPIO_SWs);
      msb_val = msb_val >> 28;										//move 4msb of switches to lsb
      lsb_val = READ_GPIO(GPIO_SWs);
      lsb_val = lsb_val >> 16;										//move 4lsb of switches to lsb
      sum = (msb_val & 0xF) + (lsb_val & 0xF);		//mask bits and add them
      if (sum < 16) {
	    WRITE_GPIO(GPIO_LEDs, sum);
      } else {
        WRITE_GPIO(GPIO_LEDs, 0x10);							//switch 5th led ON(index-1)
      }

  }
  return 0;
}
\end{lstlisting}

\subsection*{Κώδικας Assembly}
\par Επειδή το πρόγραμμα εκτελείται σε ένα SoC, πρέπει πρώτα να εκτελεστούν κάποιες 
εντολές αρχικοποίησης. Ο κώδικας που αφορά το πρόγραμμα μας βλέπουμε ότι ξεκινά στη θέση μνήμης 0x90. 
Για αποθήκευση στην μνήμη χρησιμοποιείται η στοίβα. Η κορυφή της στοίβας βρίσκεται στον καταχωρητή x2("sp"), ενώ το 
περιεχόμενο της στοίβας αυξάνεται προς τα κάτω. Ο δείκτης της στοίβας($x2/$sp) είναι κατά σύμβαση ευθυγραμμισμένος να δείχνει πάντα 
σε θέσεις μνήμης πολλαπλάσια του τεσσάρων λέξεων (quadword 16 byte aligned). Όποτε χρειάζεται νέος χώρος για προσωρινή αποθήκευση στην μνήμη, 
μειώνεται ο δείκτης στοίβας κατά ένα πολλαπλάσιο του 16, και στη συνέχεια με τη χρήση ενός θετικού offset από τον δείκτη μπορούμε να αναφερθούμε σε μια συγκεκριμένη θέση μνήμης. \\
Το extension που χρησιμοοποιείται επιτρέπει compression των κωδικών κάποιων εντολών από 4 bytes σε 2. Για λόγους "καθαρότητας" και κατανόησης του κώδικα, έχουν αφαιρεθεί οι κωδικοί 
των εντολών.

\begin{lstlisting}[ keywordstyle=\color{red}, basicstyle=\small, extendedchars=true]

Disassembly of section .text.init:

00000000 <_start>:
   0:		csrw	minstret,zero	
   4:		csrw	minstreth,zero 	
   8:		li	ra,0			#%*Αρχικοποιούνται οι καταχωρητές *)
   a:		li	sp,0
   c:		li	gp,0
   e:		li	tp,0
  10:		li	t0,0
  12:		li	t1,0
  14:		li	t2,0
  16:		li	s0,0
  18:		li	s1,0
  1a:		li	a0,0
  1c:		li	a1,0
  1e:		li	a2,0
  20:		li	a3,0
  22:		li	a4,0
  24:		li	a5,0
  26:		li	a6,0
  28:		li	a7,0
  2a:		li	s2,0
  2c:		li	s3,0
  2e:		li	s4,0
  30:		li	s5,0
  32:		li	s6,0
  34:		li	s7,0
  36:		li	s8,0
  38:		li	s9,0
  3a:		li	s10,0
  3c:		li	s11,0
  3e:		li	t3,0
  40:		li	t4,0
  42:		li	t5,0
  44:		li	t6,0
  46:		lui	t1,0x55555
  4a:		addi	t1,t1,1365 # 55555555 <_sw_int_mem_ctrl+0x51555555>
  4e:		csrw	0x7c0,t1
  52:		auipc	gp,0x3
  56:		addi	gp,gp,-1778 # 2960 <__global_pointer$>
  5a:		auipc	sp,0x3
  5e:		addi	sp,sp,262 # 3160 <_sp>
  62:		auipc	a0,0x2
  66:		addi	a0,a0,254 # 2160 <__bss_start>
  6a:		auipc	a1,0x2
  6e:		addi	a1,a1,246 # 2160 <__bss_start>
  72:		bgeu	a0,a1,80 <_start+0x80>
  76:		sw	zero,0(a0)
  7a:		addi	a0,a0,4
  7c:		bltu	a0,a1,76 <_start+0x76>
  80:		jal	ea <__libc_init_array>
  82:		li	a0,0
  84:		li	a1,0
  86:		jal	90 <main>
  88:		j	88 <_start+0x88>
  8a:		unimp
  8c:		unimp
	...

Disassembly of section .text:
#%*Εδώ ξεκινάει το πρόγραμμα μας *)
00000090 <main>:
  90:		addi	sp,sp,-16		#%*αυξάνουμε τη στοίβα κατά 4 λέξεις *)
  92:		lui	a5,0x10				#%*φορτώνουμε την τιμη 0x10 στα άνω 20 bits του a5*)
  94:		addi	a5,a5,-1		#%*αφαιρώντας 1, ο a5 παίρνει την τιμή 0xFFFF *)
  96:		sw	a5,12(sp)			#%*αποθηκεύουμε τον a5 στην στοίβα  *)
  98:		lw	a4,12(sp)			#%*δίνουμε στον a4 την τιμή που είχε ο a5(0xFFFF) *)
  9a:		lui	a5,0x80001		#%*φορτώνουμε την σταθερά στα άνω 20 bits, a5=0x80001000 *)
				#%* mem[a5+1032] = a4 , ισχύει πως 1032 = 0x408, άρα mem[0x800010408] = 0xFFFF [GPIO\_INOUT] *)
  9e:		sw	a4,1032(a5) # 80001408 <OVERLAY_END_OF_OVERLAYS+0xa0001408>
  a2:		j	ae <main+0x1e>	#%*πραγματοποιούμε jump στην διεύθυνση a4, αντιστοιχεί στην αρχή της while *)
  a4:		lui	a5,0x80001		#%*αντιστοιχεί στο "else branch", a5 = 0x80001000 *)
  a8:		li	a4,16					#%*στον a4 φορτώνουμε τη σταθερά για να ανάψει το 5ο led *)	
				#%*χρησιμοποιώντας offset ανάβουμε το 5ο led, αντιστοιχεί σε "WRITE\_GPIO(GPIO\_LEDs,0x10)" *)
  aa:		sw	a4,1028(a5) # 80001404 <OVERLAY_END_OF_OVERLAYS+0xa0001404>
  ae:		lui	a4,0x80001		#%*φορτώνουμε τη σταθερά στα άνω 20 bits, τα υπόλοιπα=0 *)
				#%* a5=mem[a4+1024], a5=mem[0x80001400], αντιστοιχεί στην msb\_val=READ\_GPIO(GPIO\_SWs) *)
  b2:		lw	a5,1024(a4) # 80001400 <OVERLAY_END_OF_OVERLAYS+0xa0001400>
  b6:		sw	a5,8(sp)			#%*αποθηκεύουμε την τιμή των διακοπτών στη στοίβα *)
  b8:		lw	a5,8(sp)			#%*την ξαναφορτώνουμε στον a5 από την στοίβα*)
  ba:		srli	a5,a5,0x1c	#%*δεξιά ολίσθηση του a5 0x1c φορές-αντοιτοιχεί στην "msb\_val $>>$ 28" *)
  bc:		sw	a5,8(sp)			#%*αποθηκεύουμε την ολισθημένη τιμή των διακοπτών στη στοίβα  *)
  be:		lw	a5,1024(a4)		#%*a5=mem[0x80001400], αντιστοιχεί στην lsb\_val=READ\_GPIO(GPIO\_SWs) *)
  c2:		sw	a5,4(sp)			#%*αποθηκεύουμε την τιμή των διακοπτών στη στοίβα *)
  c4:		lw	a5,4(sp)			#%*την ξαναφορτώνουμε στον a5 από την στοίβα(λόγω volatile) *)
  c6:		srli	a5,a5,0x10	#%*δεξιά ολίσθηση του a5 0x10 φορές-αντοιτοιχεί στην "lsb\_val $>>$ 16" *)
  c8:		sw	a5,4(sp)			#%*αποθηκεύουμε την τιμή των lsb\_val στη στοίβα *)
  ca:		lw	a5,8(sp)			#%*φορτώνουμε στον καταχωρητή a5 την τιμή της msb\_val*)
  cc:		andi	a5,a5,15		#%*κάνουμε mask τα 4 lsb του α5, αντοιτοιχεί στο "msb\_val \& 0xF"*)
  ce:		lw	a4,4(sp)			#%*φορτώνουμε στον a4 την lsb\_val από τη στοίβα *)
  d0:		andi	a4,a4,15		#%*επίσης mask τα 4 lsb, αντιστοιχεί στο "lsb\_val \& 0xF" *)
  d2:		add	a5,a5,a4			#%*προσθέτουμε τις 2 τιμές, αντιστοιχεί στην ανάθεση της sum(γραμμή 20 C) *)
  d4:		sw	a5,0(sp)			#%*αποθηκεύουμε το άθροισμα στην στοίβα *)
  d6:		lw	a4,0(sp)			#%*το φορτώνουμε στον a4 από τη στοίβα *)
  d8:		li	a5,15					#%*δίνουμε τη σταθερά 15 στον a5, θα χρησιμοποιηθεί για σύγκριση *)
				#%*Αν 15 $<$ άθροισμα(a4), τότε jump στη διεύθυνση a4 *)
  da:		bltu	a5,a4,a4 <main+0x14>
  de:		lw	a4,0(sp)			#%*φορτώνουμε την τιμή του αθροίσματος απ'τη στοίβα *)
  e0:		lui	a5,0x80001		#%*δίνουμε την τιμή x80001000 στον a5*)
				#%*και με ένα offset, ανάβουμε τα leds που πρέπει,όπως στην "WRITE\_GPIO(GPIO\_LEDs ,sum)"*)
  e4:		sw	a4,1028(a5) # 80001404 <OVERLAY_END_OF_OVERLAYS+0xa0001404>
  e8:		j	ae <main+0x1e>		#%*συνεχίζουμε το while loop *)

000000ea <__libc_init_array>:		#%*και άλλες αρχικοποιήσεις που αφορούν πιθανώς τo SoC *)
  ea:		addi	sp,sp,-16
  ec:		sw	s0,8(sp)
  ee:		sw	s2,0(sp)
  f0:		auipc	s0,0x0
  f4:		addi	s0,s0,-240 # 0 <__comrv_align_size>
  f8:		auipc	s2,0x0
  fc:		addi	s2,s2,-248 # 0 <__comrv_align_size>
 100:		sub	s2,s2,s0
 104:		sw	ra,12(sp)
 106:		sw	s1,4(sp)
 108:		srai	s2,s2,0x2
 10c:		beqz	s2,11e <__libc_init_array+0x34>
 110:		li	s1,0
 112:		lw	a5,0(s0)
 114:		addi	s1,s1,1
 116:		addi	s0,s0,4
 118:		jalr	a5
 11a:		bne	s2,s1,112 <__libc_init_array+0x28>
 11e:		auipc	s0,0x0
 122:		addi	s0,s0,-286 # 0 <__comrv_align_size>
 126:		auipc	s2,0x0
 12a:		addi	s2,s2,-294 # 0 <__comrv_align_size>
 12e:		sub	s2,s2,s0
 132:		srai	s2,s2,0x2
 136:		beqz	s2,148 <__libc_init_array+0x5e>
 13a:		li	s1,0
 13c:		lw	a5,0(s0)
 13e:		addi	s1,s1,1
 140:		addi	s0,s0,4
 142:		jalr	a5
 144:		bne	s2,s1,13c <__libc_init_array+0x52>
 148:		lw	ra,12(sp)
 14a:		lw	s0,8(sp)
 14c:		lw	s1,4(sp)
 14e:		lw	s2,0(sp)
 150:		addi	sp,sp,16
 152:		ret
\end{lstlisting}

%-----------------------------------------------------------------------------------------------------
\vspace{\baselineskip}
\vspace{\baselineskip}
\vspace{\baselineskip}
\vspace{\baselineskip}
\vspace{\baselineskip}
\vspace{\baselineskip}
\pagebreak
\section*{Ερώτημα 2}
\par
Για την υλοποίηση της καθυστέρησης, ορίζεται αυθαίρετα μια μεγάλη σταθερά η οποία καθυστερεί την εκτέλεση της ροής του προγράμματος, προσθέτοντας επιπλέον 
κύκλους ρολογιού όταν μια μεταβλητή αυξάνεται επαναληπτικά μέχρι να γίνει ίση της σταθεράς.
\par
Για την εμφάνιση της άρνησης των διακοπτών στα LEDS εξόδου, απαιτείται κάθε φορά η αντιστροφή του σημαντικότερου bit απο 1 σε 0, και από 0 σε 1 αντίστοιχα. 
Στη συγκεκριμένη υλοποίηση χρησιμοποιείται μια μεταβλητή "previous\_msb" για το λόγο αυτό.
\subsection*{Κώδικας C}

\begin{lstlisting}[language=C++, keywordstyle=\color{red}, basicstyle=\small]

// From memory-map
#define GPIO_SWs    0x80001400
#define GPIO_LEDs   0x80001404
#define GPIO_INOUT  0x80001408

#define DELAY   0x200000

//define basic read-write macros
#define READ_GPIO(addr) (*(volatile unsigned*)addr)
#define WRITE_GPIO(addr, value) { (*(volatile unsigned *)addr) = (value);}

int main(void) 
{
    volatile unsigned ddr_value=0xFFFF, switches_val, temp; 
    volatile unsigned ones_sum = 0, previous_msb = 0;
    volatile unsigned i, timer;
    WRITE_GPIO(GPIO_INOUT, ddr_value);				//set leds as output
    previous_msb = READ_GPIO(GPIO_SWs); 			//read switches
    previous_msb &= 0x80000000;								//keep msb for comparison
    while (1) {
        do {															//continue reading untill msb changes
        switches_val = READ_GPIO(GPIO_SWs);
        ;
        } while ((switches_val & 0x80000000) == previous_msb);	
        previous_msb = switches_val & 0x80000000;			//Update msbit 
        switches_val = switches_val >> 16;
        switches_val &= 0xFFFF;
        switches_val ^= 0xFFFF; 											//flip bits using XOR
        temp = switches_val;
        ones_sum = 0;
        for (i=0; i<16; i++) {												//shift and count 1-bits
            if (temp & 0x1) {
                ones_sum ++;
            }
            temp =  temp >> 1;
        }

        for (i=1; i<=ones_sum; i++) {							//switch ON and OFF with delay
            timer = 0;
            WRITE_GPIO(GPIO_LEDs, switches_val);
            while(timer < DELAY) {
                timer++;
            }
            timer = 0;
            WRITE_GPIO(GPIO_LEDs, 0);
            while(timer < DELAY) {
                timer++;
            }
        }
    }
    return 0;
}
\end{lstlisting}

\subsection*{Κώδικας Assembly}

\begin{lstlisting}[ keywordstyle=\color{red}, basicstyle=\small]
Disassembly of section .text.init:

00000000 <_start>:		#%*Αρχικοποιούνται οι καταχωρητές *)
   0:	csrw	minstret,zero
   4:	csrw	minstreth,zero
   8:	li	ra,0
   a:	li	sp,0
   c:	li	gp,0
   e:	li	tp,0
  10:	li	t0,0
  12:	li	t1,0
  14:	li	t2,0
  16:	li	s0,0
  18:	li	s1,0
  1a:	li	a0,0
  1c:	li	a1,0
  1e:	li	a2,0
  20:	li	a3,0
  22:	li	a4,0
  24:	li	a5,0
  26:	li	a6,0
  28:	li	a7,0
  2a:	li	s2,0
  2c:	li	s3,0
  2e:	li	s4,0
  30:	li	s5,0
  32:	li	s6,0
  34:	li	s7,0
  36:	li	s8,0
  38:	li	s9,0
  3a:	li	s10,0
  3c:	li	s11,0
  3e:	li	t3,0
  40:	li	t4,0
  42:	li	t5,0
  44:	li	t6,0
  46:	lui	t1,0x55555
  4a:	addi	t1,t1,1365 # 55555555 <_sw_int_mem_ctrl+0x51555555>
  4e:	csrw	0x7c0,t1
  52:	auipc	gp,0x3
  56:	addi	gp,gp,-1650 # 29e0 <__global_pointer$>
  5a:	auipc	sp,0x3
  5e:	addi	sp,sp,390 # 31e0 <_sp>
  62:	auipc	a0,0x2
  66:	addi	a0,a0,382 # 21e0 <__bss_start>
  6a:	auipc	a1,0x2
  6e:	addi	a1,a1,374 # 21e0 <__bss_start>
  72:	bgeu	a0,a1,80 <_start+0x80>
  76:	sw	zero,0(a0)
  7a:	addi	a0,a0,4
  7c:	bltu	a0,a1,76 <_start+0x76>
  80:	jal	16a <__libc_init_array>
  82:	li	a0,0
  84:	li	a1,0
  86:	jal	90 <main>
  88:	j	88 <_start+0x88>
  8a:	unimp
  8c:	unimp
	...

Disassembly of section .text:

00000090 <main>:
  90:	addi	sp,sp,-32		#%*αυξάνουμε τη στοίβα κατά 8 λέξεις *)	
  92:	lui	a5,0x10				#%*φορτώνουμε τη σταθερά στα άνω 20 bits του a5, a5= 0x00010000*)	
  94:	addi	a5,a5,-1		#%*μειώνοντας 1, γίνεται a5=0x0000FFFF *)	
  96:	sw	a5,28(sp)			#%*αποθηκεύουμε την τιμή του a5 στη στοίβα *)	
  98:	sw	zero,16(sp)		#%*αποθηκεύουμε την τιμή 0 στη στοίβα *)	
  9a:	sw	zero,12(sp)		#%*αποθηκεύουμε την τιμή 0 στη στοίβα *)	
  9c:	lw	a4,28(sp)			#%*φορτώνουμε την τιμή 0x0000FFFF από τη στοίβα στον a5 *)	
  9e:	lui	a5,0x80001		#%*φορτώνουμε τη σταθερά στα άνω 20 bits του a5, a5= 0x80001000*)	
			#%*με ένα offset 0x408 από τον a5, θέτουμε τα leds σαν έξοδο στη διεύθυνση 0x80001408*)	
  a2:	sw	a4,1032(a5) # 80001408 <OVERLAY_END_OF_OVERLAYS+0xa0001408>
  a6:	lw	a5,1024(a5)		#%*διαβάζουμε τους διακόπτες απτή θέση μνήμης 0x80001400 *)	
  aa:	sw	a5,12(sp)			#%*αποθηκεύουμε την τιμή των διακοπτών στη στοίβα *)	
  ac:	lw	a5,12(sp)			#%*ξανα φορτώνουμε την τιμή των διακοπτών από τη στοίβα *)	
  ae:	lui	a4,0x80000		#%*φορτώνουμε τη σταθερά στα άνω 20 bits του a4, a4= 0x80000000*)	
  b2:	and	a5,a5,a4			#%*κάνουμε mask την τιμή των διακοπτών με τη σταθερά 0x80000000*)	
  b4:	sw	a5,12(sp)			#%* αποθηκεύουμε την τιμή του σημαντικότερου bit στη στοίβα*)	
  b6:	j	e6 <main+0x56>	#%*κάνουμε jump στη διεύθυνση 0xe6 *)	
  b8:	lw	a5,20(sp)			#%*φορτώνουμε την μεταβλητή temp από τη στοίβα *)	
  ba:	srli	a5,a5,0x1		#%*την ολισθαίνουμε δεξιά κατά 1*)	
  bc:	sw	a5,20(sp)			#%*ενημερώνουμε την τιμή της στην στοίβα*)	
  be:	lw	a5,8(sp)			#%*φορτώνουμε τον μετρητή i από τη στοίβα *)	
  c0:	addi	a5,a5,1			#%*τον αυξάνουμε κατά 1*)	
  c2:	sw	a5,8(sp)			#%*τον ενημερώνουμε στην στοίβα*)	
  c4:	lw	a4,8(sp)			#%*φορτώνουμε τον μετρητή i του for-loop από τη στοίβα *)	
  c6:	li	a5,15					#%*θέτουμε τον a=15*)	
			#%*αν i$>$15 σπάμε το loop με άλμα στην θέση μνήμης da *)	
  c8:	blt	a5,a4,da <main+0x4a>
  cc:	lw	a5,20(sp)			#%*φορτώνουμε την μεταβλητή temp από τη στοίβα*)	
  ce:	andi	a5,a5,1			#%*κάνουμε mask κρατώντας μόνο το lsb*)	
			#%*αν ισούται με 0, κάνουμε άλμα και δεν αυξάνουμε την ones\_sum*)	
  d0:	beqz	a5,b8 <main+0x28>
  d2:	lw	a5,16(sp)			#%*αλλιώς φορτώνουμε από τη στοίβα την ones\_sum*)	
  d4:	addi	a5,a5,1			#%*και την αυξάνουμε κατά 1 επειδή βρήκαμε άσσο*)	
  d6:	sw	a5,16(sp)			#%*ενημερώνουμε την τιμή της στη στοίβα*)	
  d8:	j	b8 <main+0x28>	#%*κάνουμε άλμα στη θέση b8, όπου γίνεται η ολίσθηση *)	
  da:	li	a5,1					#%*αρχικοποιούμε τον μετρητή i=1*)	
  dc:	sw	a5,8(sp)			#%*τον ενημερώνουμε στην στοίβα*)	
  de:	lw	a4,8(sp)			#%*φορτώνουμε τον i από την στοίβα(volatile)*)	
  e0:	lw	a5,16(sp)			#%*θέτουμε τον a5=ones\_sum για την σύγκριση του for\-loop*)	
			#%*αν ones\_sum$>=$ i κάνουμε άλμα στην θέση 128*)	
  e2:	bgeu	a5,a4,128 <main+0x98>
  e6:	lui	a5,0x80001		#%*φορτώνουμε τη σταθερά στα άνω 20 bits του a5, a5= 0x80001000 *)
			#%*διαβάζουμε τους διακόπτες απτή θέση μνήμης 0x80001400 *)	
  ea:	lw	a5,1024(a5) # 80001400 <OVERLAY_END_OF_OVERLAYS+0xa0001400>
  ee:	sw	a5,24(sp)			#%*αποθηκεύουμε την τιμή των διακοπτών στη στοίβα *)	
  f0:	lw	a5,24(sp)			#%*ξανα φορτώνουμε την τιμή των διακοπτών από τη στοίβα(volatile)  *)	
  f2:	lui	a4,0x80000		#%*φορτώνουμε τη σταθερά στα άνω 20 bits του a4, a4= 0x80000000*)	
  f6:	and	a5,a5,a4			#%*κάνουμε mask τη τιμή των διακοπτών κρατώντας το σημαντικότερο bit*)	
  f8:	lw	a4,12(sp)			#%*φορτώνουμε από τη στοίβα τη τιμή του προηγούμενου σημαντικότερου bit*)	
			#%*αν είναι ίσες οι τιμές, jump στην θέση μνήμης 0xE6 και συνέχιση του do while()*)	
  fa:	beq	a5,a4,e6 <main+0x56>	#%*αλλιώς είχαμε αλλαγή του σημαντικότερου bit*)
  fe:	lw	a5,24(sp)			#%* φόρτωση της τιμής των διακοπτών από τη στοίβα *)	
 100:	lui	a4,0x80000		#%*φορτώνουμε τη σταθερά στα άνω 20 bits του a4, a4= 0x80000000*)	
 104:	and	a5,a5,a4			#%*κάνουμε mask τη τιμή των διακοπτών κρατώντας το σημαντικότερο bit *)	
 106:	sw	a5,12(sp)			#%*και ενημερώνουμε την τιμή του προηγούμενου σημαντικότερου bit στη στοίβα *)	
 108:	lw	a5,24(sp)			#%*φόρτωση της τιμής των διακοπτών από τη στοίβα  *)	
			#%*κάνουμε δεξιά ολίσθηση την τιμή κατά 16 φορές, για να τη φέρουμε στα 16 lsb *)	
 10a:	srli	a5,a5,0x10
 10c:	sw	a5,24(sp)			#%*αποθηκεύουμε την ολισθημένη τιμή στη στοίβα*)	
 10e:	lw	a4,24(sp)			#%*την φορτώνουμε στον a4 από τη στοίβα(volatile) *)	
 110:	lui	a5,0x10				#%* φορτώνουμε τη σταθερά στα άνω 20 bits του a5, a5= 0x00010000*)	
 112:	addi	a5,a5,-1		#%* αφαιρώντας 1, ο a5 γίνεται 0x0000FFFF*)	
 114:	and	a4,a4,a5			#%* κάνουμε mask τα 16 lsb *)	
 116:	sw	a4,24(sp)			#%* τα αποθηκεύουμε στη στοίβα*)	
 118:	lw	a4,24(sp)			#%* τα φορτώνουμε από τη στοίβα*)	
 11a:	xor	a5,a5,a4			#%* αντιστρέφουμε τα 16 lsb, και τα αποθηκεόυμε στον a5*)	
 11c:	sw	a5,24(sp)			#%* αποθηκεύουμε την αντεστραμμένη τιμή στη στοίβα*)	
 11e:	lw	a5,24(sp)			#%* φορτώνουμε την αντεστραμμένη τιμή από τη στοίβα(volatile)*)	
 120:	sw	a5,20(sp)			#%* την αποθηκεύουμε στη θέση μνήμης mem[sp+20]*)	
 122:	sw	zero,16(sp)		#%*αποθηκεύουμε το 0 στη θέση μνήμης mem[sp+16]*)	
 124:	sw	zero,8(sp)		#%*αποθηκεύουμε το 0 στη θέση μνήμης mem[sp+8]-μετρητής loop *)	
 126:	j	c4 <main+0x34>	#%*πραγματοποιούμε άλμα στη θέση μνήμης c4*)	
 128:	sw	zero,4(sp)		#%*αρχικοποιούμε την mem[sp+4] = 0, θέση μεταβλητής timer*)	
 12a:	lw	a4,24(sp)			#%*φορτώνουμε την τιμή των διακοπτών στον a4*)	
 12c:	lui	a5,0x80001		#%*φορτώνουμε την τιμή 0x80001000 *)	
			#%*και με offset ανάβουμε τα leds, mem[0x80001404]=switches\_val *)	
 130:	sw	a4,1028(a5) # 80001404 <OVERLAY_END_OF_OVERLAYS+0xa0001404>
 134:	lw	a4,4(sp)			#%*φορτώνουμε την τιμή του timer*)	
 136:	lui	a5,0x200			#%*φορτώνουμε την σταθερά καθυστέρησης *)	
			#%*Αν τελειώσει η καθυστέρηση, άλμα στην θέση 146*)	
 13a:	bge	a4,a5,146 <main+0xb6>
 13e:	lw	a5,4(sp)			#%*φορτώνουμε την τιμή του timer*)	
 140:	addi	a5,a5,1			#%*αυξάνουμε κατά 1*)	
 142:	sw	a5,4(sp)			#%*την ενημερώνουμε στην στοίβα *)	
 144:	j	134 <main+0xa4>	#%*και συνεχίζουμε το loop καθυστέρησης*)	
 146:	sw	zero,4(sp)		#%*έδώ τελείωσε η καθυστέρηηση, αρχικοποιούμε πάλι τον timer *)	
 148:	lui	a5,0x80001		#%*a5 = 0x80001000*)	
			#%*θέτουμε τη θέση μνήμης των led=0*)	
 14c:	sw	zero,1028(a5) # 80001404 <OVERLAY_END_OF_OVERLAYS+0xa0001404>
 150:	lw	a4,4(sp)			#%*φορτώνουμε τον timer *)	
 152:	lui	a5,0x200			#%*θέτουμε την σταθερά της σύγκρισης *)	
			#%*Αν τελειώσει η καθυστέρηση, άλμα στην θέση 162*)	
 156:	bge	a4,a5,162 <main+0xd2>
 15a:	lw	a5,4(sp)			#%*αλλιώς φορτώνουμε την τιμή του timer*)	
 15c:	addi	a5,a5,1			#%*τον αυξάνουμε κατά 1*)	
 15e:	sw	a5,4(sp)			#%*και τον ενημερώνουμε στην στοίβα*)	
 160:	j	150 <main+0xc0>	#%*άλμα και συνέχιση του loop καθυστέρησης*)	
 162:	lw	a5,8(sp)			#%*φορτώνουμε τον μετρητή του εξωτερικού loop*)	
 164:	addi	a5,a5,1			#%*τον αυξάνουμε κατά 1*)	
 166:	sw	a5,8(sp)			#%*και τον ενημερώνουμε στην στοίβα*)	
 168:	j	de <main+0x4e>	#%*άλμα στη θέση de *)	

0000016a <__libc_init_array>:
 16a:	addi	sp,sp,-16
 16c:	sw	s0,8(sp)
 16e:	sw	s2,0(sp)
 170:	auipc	s0,0x0
 174:	addi	s0,s0,-368 # 0 <__comrv_align_size>
 178:	auipc	s2,0x0
 17c:	addi	s2,s2,-376 # 0 <__comrv_align_size>
 180:	sub	s2,s2,s0
 184:	sw	ra,12(sp)
 186:	sw	s1,4(sp)
 188:	srai	s2,s2,0x2
 18c:	beqz	s2,19e <__libc_init_array+0x34>
 190:	li	s1,0
 192:	lw	a5,0(s0)
 194:	addi	s1,s1,1
 196:	addi	s0,s0,4
 198:	jalr	a5
 19a:	bne	s2,s1,192 <__libc_init_array+0x28>
 19e:	auipc	s0,0x0
 1a2:	addi	s0,s0,-414 # 0 <__comrv_align_size>
 1a6:	auipc	s2,0x0
 1aa:	addi	s2,s2,-422 # 0 <__comrv_align_size>
 1ae:	sub	s2,s2,s0
 1b2:	srai	s2,s2,0x2
 1b6:	beqz	s2,1c8 <__libc_init_array+0x5e>
 1ba:	li	s1,0
 1bc:	lw	a5,0(s0)
 1be:	addi	s1,s1,1
 1c0:	addi	s0,s0,4
 1c2:	jalr	a5
 1c4:	bne	s2,s1,1bc <__libc_init_array+0x52>
 1c8:	lw	ra,12(sp)
 1ca:	lw	s0,8(sp)
 1cc:	lw	s1,4(sp)
 1ce:	lw	s2,0(sp)
 1d0:	addi	sp,sp,16
 1d2:	ret

\end{lstlisting}



\end{document}